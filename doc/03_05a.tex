\subsection{Inherit the Earth notes}
In order to run MacOS X wyrmkeep rerelease of the game you will need to copy
over data from the CD to hard drive. If you're on PC then consult

\url{http://www.scummvm.org/documentation.php?view=maccd-howto}

Although it talks about SCUMM games, it describes HFVExplorer utility. Note
that you will need to put speech data "Inherit the Earth Voices" in same
directory as game data which is stored in

\begin{verbatim}
  Inherit the Earth.app/Contents/Resources
\end{verbatim}

For old Mac OS9 release you will need to compile files in MacBinary format,
i.e. they should have both resource and data forks. Just copy all 'ITE *' files.


\subsection{Gobliiins notes}
CD version of Gobliiins contains one big audio track which needs to be ripped
and copied into game directory. See section \ref{sect-compressing-audiofiles}.


\subsection{Maniac Mansion NES notes}
Supported versions are English USA (E), French (F), Swedish (SW) and 
European (U). ScummVM requires just PRG section to run and not whole ROM.

In order to get the game working, you will have to strip out the first
16 bytes from the ROM you are trying to work with. Any hex editor will work 
as long as you are able to copy/paste.  After you open the ROM with the 
hex editor, copy everything from second row (17th byte) to the end. After
you do this, paste it to a new hex file. Give the new file name 
"Maniac Mansion (XX).prg" where XX depends on version you are  working 
with (E, F, SW, or U).  The final size should be exactly 262144 bytes.

If you add game manually make sure that platform is set to NES.

Most common mistakes which prevent game from running:

\begin{itemize}
  \item Bad file
  \item ROM extracted with the 0.7.0 tools
  \item You try to feed ScummVM with the FULL rom and not just the PRG section.
\end{itemize}

Also it is possible to extract separate LFL files from PRG section. Use
extract\_mm\_nes utility from tools package.


\subsection{Commodore64 games notes}
Both Maniac Mansion and Zak McKracken run but Maniac Mansion is not yet
playable. Either use extract\_mm\_c64 (but then game will not be autodetected)
or name D64 disks as "maniac1.d64", "maniac2.d64" and "zak1.d64", "zak2.d64"
respectively. If you add the game manually, make sure that platform is set to
Commodore64.
