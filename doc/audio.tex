
%%% Local Variables: 
%%% mode: latex
%%% TeX-master: "readme.tex"
%%% End: 


\section{Music and Sound} \label{sect-music-and-sound}

By default, on most operating systems, ScummVM will automatically use Adlib
emulation. MIDI may not be available on all operating systems or may need
manual configuration. If you ARE using MIDI, you have several different
choices of output, depending on your operating system and configuration.

\begin{tabular}[h]{ll}
  null       & Null output. Don't play any music.\\
             & \\
  adlib      & Uses internal Adlib Emulation (default)\\
  fluidsynth & Uses FluidSynth MIDI Emulation\\
  mt32       & Uses internal MT-32 Emulation\\
  pcjr       & Uses internal PCjr Emulation \\
  pcspk      & Uses internal PC Speaker Emulation\\
  towns      & Uses FM-TOWNS YM2612 Emulation\\
             & \\
  alsa       & Output using ALSA sequencer device. See below.\\
  core       & CoreAudio sound, for Mac OS X users.\\
  coremidi   & CoreMIDI sound, for Mac OS X users. Use only if you have a hardware MIDI synthesizer.\\
  qt         & Quicktime sound, for Macintosh users.\\
  seq        & Uses /dev/sequencer for MIDI, *nix users. See below.\\
  windows    & Windows MIDI. Uses built-in sequencer, for Windows users\\
\end{tabular}

To select a sound driver, pass its name via the '-e' option to scummvm,
for example:
\begin{verbatim}
   scummvm -eadlib monkey2
\end{verbatim}


\input {audio-adlib.tex}

%%% Local Variables: 
%%% mode: latex
%%% TeX-master: "readme.tex"
%%% End: 


\subsection{Playing sound with FluidSynth MIDI emulation}

If ScummVM was build with libfluidsynth support it will be able to play MIDI
music through the FluidSynth driver. You will have to specify a SoundFont to
use, however.

Since the default output volume from FluidSynth can be fairly low, ScummVM will
set the gain by default to get a stronger signal. This can be further adjusted
using the --midi-gain command-line option, or the "midi\_gain" config file
setting.

The setting can take any value from 0 through 1000, with the default being 100.
(This corresponds to FluidSynth's gain settings of 0.0 through 10.0, which are
presumably measured in decibel.)

\textbf{NOTE:} The processor requirements for FluidSynth can be fairly high in
some cases. A fast CPU is recommended.

\subsection{Playing sound with MT-32 emulation}

Some games which contain MIDI music data also have improved tracks designed
for the MT-32 sound module. ScummVM can now emulate this device, however you
must provide original MT-32 ROMs to make it work:

MT32\_PCM.ROM     - IC21 (512KB)\\
MT32\_CONTROL.ROM - IC26 (32KB) and IC27 (32KB), interleaved byte-wise\\

Place these ROMs in the game directory, in your extrapath, or in the directory
where your ScummVM executable resides.

You don't need to specify --native-mt32 with this driver, as it automatically
gets turned on.

\textbf{NOTE:} The processor requirements for the emulator are quite high; a fast CPU is
 strongly recommended.


%%% Local Variables: 
%%% mode: latex
%%% TeX-master: "readme.tex"
%%% End: 


%%% Local Variables: 
%%% mode: latex
%%% TeX-master: "readme.tex"
%%% End: 


\subsection{Playing sound with MIDI emulation}

Some games (such as Sam \& Max) only contain MIDI music data.  This once
prevented music for these games from working on platforms that do not support
MIDI, or soundcards that do not provide MIDI drivers (e.g, many soundcards will
not play MIDI under Linux). ScummVM can now emulate MIDI mode using sampled
waves and Adlib, FluidSynth MIDI emulation or MT-32 emulation using the
-eadlib, -efluidsynth or -emt32 options respectively.  However, if you are
capable of using native MIDI, we recommend using one of the MIDI modes below
for best sound.


%%% Local Variables: 
%%% mode: latex
%%% TeX-master: "readme.tex"
%%% End: 


\subsection{Playing sound with Native MIDI}
% FIXME: Hardcoding < and > here produces wrong output

Use the appropriate -e<mode> command line option from the list above to
select your preferred MIDI device. For example, if you wish to use the
Windows MIDI driver, use the -ewindows option.


\subsubsection{Using MIDI options to customize Native MIDI output}

ScummVM supports a variety of MIDI modes, depending on the capabilities
of your MIDI device.

If --native-mt32 is specified, ScummVM will treat your device as a real
MT-32. Because the instrument mappings and system exclusive commands of
the MT-32 vary from those of General MIDI devices, you should only
enable this option if you are using an actual Roland MT-32, LAPC-I, CM-64,
CM-32L, CM-500, or GS device with an MT-32 map.

If --enable-gs is specified, ScummVM will initialize your GS-compatible
device with settings that mimic the MT-32's reverb, (lack of) chorus,
pitch bend sensitivity, etc. If it is specified in conjunction with
--native-mt32, ScummVM will select the MT-32-compatible map and drumset on
your GS device. This setting works better than default GM or GS emulation
with games that do not have custom instrument mappings (Loom and Monkey1).
You should only specify both settings if you are using a GS device that
has an MT-32 map, such as an SC-55, SC-88, SC-88 Pro, SC-8820, SC-8850, etc.
Please note that --enable-gs is automatically disabled in both DOTT and
Samnmax, since they use General MIDI natively.

If neither of the above settings is enabled, ScummVM will initialize your
device in General MIDI mode and use GM emulation in games with MT-32
soundtracks.

Some games contain sound effects that are exclusive to the Adlib soundtrack.
For these games, you may wish to specify --multi-midi in order to combine
MIDI music with Adlib sound effects.

\input {audio-midi-sequencer.tex}
\input {audio-compression.tex}
\input {audio-samplerate.tex}
