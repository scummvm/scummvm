
%%% Local Variables: 
%%% mode: latex
%%% TeX-master: "readme.tex"
%%% End: 


\subsection{Playing sound with Native MIDI}
% FIXME: Hardcoding < and > here produces wrong output

Use the appropriate -e<mode> command line option from the list above to
select your preferred MIDI device. For example, if you wish to use the
Windows MIDI driver, use the -ewindows option.


\subsubsection{Using MIDI options to customize Native MIDI output}

ScummVM supports a variety of MIDI modes, depending on the capabilities
of your MIDI device.

If --native-mt32 is specified, ScummVM will treat your device as a real
MT-32. Because the instrument mappings and system exclusive commands of
the MT-32 vary from those of General MIDI devices, you should only
enable this option if you are using an actual Roland MT-32, LAPC-I, CM-64,
CM-32L, CM-500, or GS device with an MT-32 map.

If --enable-gs is specified, ScummVM will initialize your GS-compatible
device with settings that mimic the MT-32's reverb, (lack of) chorus,
pitch bend sensitivity, etc. If it is specified in conjunction with
--native-mt32, ScummVM will select the MT-32-compatible map and drumset on
your GS device. This setting works better than default GM or GS emulation
with games that do not have custom instrument mappings (Loom and Monkey1).
You should only specify both settings if you are using a GS device that
has an MT-32 map, such as an SC-55, SC-88, SC-88 Pro, SC-8820, SC-8850, etc.
Please note that --enable-gs is automatically disabled in both DOTT and
Samnmax, since they use General MIDI natively.

If neither of the above settings is enabled, ScummVM will initialize your
device in General MIDI mode and use GM emulation in games with MT-32
soundtracks.

Some games contain sound effects that are exclusive to the Adlib soundtrack.
For these games, you may wish to specify --multi-midi in order to combine
MIDI music with Adlib sound effects.
