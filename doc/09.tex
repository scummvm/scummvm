
%%% Local Variables: 
%%% mode: latex
%%% TeX-master: "readme"
%%% End: 

\section{Compiling}

You need SDL-1.2.2 or newer (older versions may work, but are unsupported), and
a supported compiler. Several compilers, including GCC, mingw and Microsoft
Visual C++ are supported. If you wish to use MP3-compressed CD tracks or
.SOU files, you will need to install the MAD library and define
USE\_MAD. Tools for compressing .SOU files to .SO3 files can be
found in the 'tools' CVS module, or in the 'scummvm-tools' package.

On Win9x/NT/XP you can define USE\_WINDBG and attach WinDbg to browse debug 
messages (see \url{http://www.sysinternals.com/ntw2k/freeware/debugview.shtml}).

\subsection{GCC}
  \begin{itemize}
  \item Type \texttt{./configure}
  \item Type \texttt{make} (or \texttt{gmake}, or \texttt{gnumake}, depending
        on what GNU make is called on your system) and hopefully ScummVM will
        compile for you.
  \end{itemize}
\subsection{MingW -  Windows 5/98/ME/NT/2000/XP/2003}
  \begin{itemize}
  \item Open Makefile.mingw, alter SDL paths and choose compiling
    options.
  \item Type make -f Makefile.mingw,  hopefully ScummVM will compile for you.
  \end{itemize}
\subsection{Microsoft Visual C++ 6.0}
  \begin{itemize}
  \item Open the workspace, scummwm.dsw
  \item Enter the path to the needed libraries and includes in
    Tools|Options|Directories
  \item Now it should compile successfully. 
  \end{itemize}
\subsection{Microsoft Visual C++ 7.0}
  \begin{itemize}
  \item Open the solution file scummwm.sln
  \item Enter the path to the needed libraries and includes in
    Tools|Options|Directories
  \item Now it should compile successfully. 
  \end{itemize}
\subsection{PocketPC Windows CE}
  \begin{itemize}
  \item Download the SDLAudio library:\\
    \url{http://arisme.free.fr/PocketScumm/sources/SDLAudio-1.2.3-src.zip}
  \item Open and compile the SDLAudio WCEBuild/WCEBuild workspace in
    EVC++
  \item Open the ScummVM wince/PocketScumm workspace
  \item Enter the SDLAudio directory to your includes path
  \item Enter the compiled SDLAudio.lib to your link libraries list
  \item Now it should compile successfully
  \end{itemize}
\subsection{Debian GNU/Linux}
  \begin{itemize}
  \item Install the packages 'build-essential', 'fakeroot', 'debhelper',
          and 'libsdl1.2-dev' on your system.
  \item nstall any of these packages (optional): 'libvorbis-dev' (for Ogg
          Vorbis support), 'libasound2-dev' (for ALSA sequencer support),
          'libmad0-dev' (for MAD MP3 support), 'zlib1g-dev' (for compressed
          saves support).
  \item Run 'make deb'
  \item Finally run 'dpkg -i ../scummvm-cvs*deb', and you're done.
  \end{itemize}
\subsection{Mac OS X}
\begin{itemize}
\item Make sure you have the developer tools installed.
\item The SDL developer package for OS X available on the SDL web site is
      \textit{not} suitable. Rather, you require a unix-style build of SDL. One
      way to get that is to install SDL via Fink (\url{http://fink.sf.net}).
      Alternatively you could compile SDL manually from source using its
      unix build system (\texttt{configure \&\& make}).
\item Type \texttt{./configure} in the ScummVM directory
\item You can now type \texttt{make} to create a command line binary.
\item To get a version you can run from Finder, type \texttt{make bundle} which
      will create ScummVM.app (this only works if you installed SDL
      etc. via Fink and into /sw. If you have installed SDL in another
      way, you'll have to edit the Makefile).
\end{itemize}
